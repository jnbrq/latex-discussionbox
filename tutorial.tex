% Canberk Sönmez

\documentclass{article}

\usepackage[margin=2cm]{geometry}
\usepackage{discussionbox}

\renewcommand{\familydefault}{\sfdefault}

\DefinePerson{Intro}{black}
\DefinePerson{Basics}{black}
\DefinePerson{Communication}{black}

\begin{document}

\begin{discussionbox}[Tutorial (D\dbox{nextid})]
    \Intro{
        DiscussionBox provides a forum-like typesetting for discussions taking place on \LaTeX{} document sharing platforms like Overleaf.
        Compared to review mechanisms provided by such platforms, DiscussionBox enables non-volatile and structured conversations, somewhat similar to e-mail.
        Unlike e-mail, however, you can benefit from all of the powerful \LaTeX{} features!
    }

    \begin{dboxmsg}{Basics}
        The fundamental component of the DiscussionBox package is, well, a \texttt{discussionbox}:
        \begin{verbatim}
    % discussion boxes are declared as an environment:
    \begin{discussionbox}[Discussion \dbox{nextid}] % title is optional
        % you can use \dbox{nextid} to enumerate titles
        % the default title is "Discussion Box #1"

        % messages...
    \end{discussionbox}
        \end{verbatim}

        Please check \texttt{discussionbox.sty} file to learn about the customization opportunities.
    \end{dboxmsg}

    \begin{dboxmsg}{Basics}
        A discussion box is merely a container for \textit{message}s and a message is typed by a \textit{person}.

        \begin{verbatim}
    % in the preamble, define a person:
    \DefinePerson{Canberk}{blue} % \DefinePerson{name}{color}
    
    % within the document:

    \begin{discussionbox}

        % messages using a simple command:
        \Canberk{
            Your message here...
        }

        % or, defined as an environment: (which might work better in some cases)
        \begin{dboxmsg}{Canberk} % \begin{dboxmsg}{person name}...\end{dboxmsg}
            Your message here...
        \end{dboxmsg}

    \end{discussionbox}
        \end{verbatim}
    
    \noindent The name of the author is shown in the beginning of the message, and the entire message colored with the color of the author.
    \end{dboxmsg}

    \Communication{\LabelMsg{msg:msg1}
        The main idea of a DiscussionBox is to enable effective communication over.
        For that purpose, it provides four important features:
        (1) \textit{message addresses}, (2) \textit{references}, (3) \textit{replies} and (4) \textit{see-messages}.
    }
    \begin{dboxmsg}{Communication}
        Each message in a discussion box is uniquely identified by an address.
        For example, the address of this one \dbox{msgaddr}.
        D$x$ stands for the discussion with number $x$ and the rest of the address identifies the message within a discussion.
        Addresses appear right next to the author name and formatted as superscript.
        You can use the address of a message to refer to it.
        Similar to other \LaTeX{} objects that can be referenced, you need to first label the message:

        \begin{verbatim}
    % use \LabelMsg{label} to label a message
    \Canberk{Hello, I say things .... \LabelMsg{canberk:things}}

    % later...

    \Mehmet{Note what Canberk says \ref{canberk:things}, he says some serious stuff.}
        \end{verbatim}
        
        \noindent Let's reference the previous message: \ref{msg:msg1}!
        You can click on the reference to navigate to the message.
        \textbf{Note that, on Overleaf, you must use ``Browser'' instead of ``Overleaf'' as the PDF viewer for hyperlinks to work.}
        \LabelMsg{msg:msg2}

        \begin{dboxmsg}{Communication}
            A message can be a reply to another message.
            For example, this message (\dbox{msgaddr}) replies to \ref{msg:msg2}.
            Message addresses reflect reply-to relations.
            If a message has address D$x$.$y$.$z$.$n$, it is the $n$\textsuperscript{th} reply to D$x$.$y$.$z$.
        \end{dboxmsg}
        \begin{dboxmsg}{Communication}
            For instance, the last character of the address of this message is 2.
            A reply message is nested within the message that it replies to:
            \begin{verbatim}
    \Mehmet{I think Switzerland is the most prominent economy in Europe.
            \Canberk{I disagree, it is Germany.}
    }

    % works the same way with environments
            \end{verbatim}
        \end{dboxmsg}
    \end{dboxmsg}

    \begin{dboxmsg}{Communication}
        Finally, DiscussionBox provides a simple, in-document mailbox mechanism to notify people of new messages, namely see-message:

        \begin{verbatim}
            % list the unread messages for someone:
            \ListSeeMsgs{Canberk} % usage: \ListSeeMsgs{person}

            % mark a message unread for someone:
            % usage: \SeeMsg{person}
            \Mehmet{
                \SeeMsg{Canberk} should see this!
            }
        \end{verbatim}

        The person who read the message marked to be seen can unmark it by removing \verb|\SeeMsg{...}|.
    \end{dboxmsg}
\end{discussionbox}

\end{document}
